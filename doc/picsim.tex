%
% $Id: picsim.tex,v 1.17 2013/03/15 17:33:12 patrick Exp $
%
%
% Copyright (c) 2000-2011 Patrick Guio <patrick.guio@gmail.com>
% All Rights Reserved.
%
% This program is free software; you can redistribute it and/or modify it
% under the terms of the GNU General Public License as published by the
% Free Software Foundation; either version 2.  of the License, or (at your
% option) any later version.
%
% This program is distributed in the hope that it will be useful, but
% WITHOUT ANY WARRANTY; without even the implied warranty of
% MERCHANTABILITY or FITNESS FOR A PARTICULAR PURPOSE.  See the GNU General
% Public License for more details.
%
% You should have received a copy of the GNU General Public License
% along with this program. If not, see <http://www.gnu.org/licenses/>.

\documentclass[10pt,a4paper]{article}
\usepackage{times}
\usepackage{fullpage}
\usepackage{amsmath}
\usepackage[round]{natbib}
\usepackage[latin1,utf8]{inputenc}
\usepackage{graphicx}
\usepackage{fancybox}
\usepackage{tabularx}
\usepackage{needspace}
\usepackage{esdiff}
\usepackage{physics}
\usepackage{xspace}
\usepackage{html}
\usepackage{version}

%\parindent 0cm

\newlength{\firstcol}
\setlength{\firstcol}{.35\textwidth}
\newlength{\secondcol}
\setlength{\secondcol}{.65\textwidth}

\newlength{\mylength}

\newenvironment{falign}%
{\setlength{\fboxsep}{15pt}
\setlength{\fboxrule}{0.5pt}
\setlength{\mylength}{\textwidth}
\addtolength{\mylength}{-2\fboxsep}
\addtolength{\mylength}{-2\fboxrule}
\Sbox
\minipage{\mylength}%
  \setlength{\abovedisplayskip}{-2\lineskip}
	\setlength{\belowdisplayskip}{-2\lineskip}
\align}%
{\endalign\endminipage\endSbox
\[\fbox{\TheSbox}\]}

\newcommand*{\Eq}[1]{Eq.~(#1)}
\newcommand*{\Eqs}[1]{Eqs.~(#1)}

\newcommand*{\matlab}{\textsf{Matlab}\xspace}
\newcommand*{\mgfas}{\textsf{Mgfas}\xspace}
\newcommand*{\mudpack}{\textsf{Mudpack}\xspace}
\newcommand*{\mudfas}{\textsf{Mudfas}\xspace}
\newcommand*{\picsim}{\textsf{PicSim}\xspace}

\newcommand{\CIC}{\renewcommand{\CIC}{CIC\xspace}Cloud-In-Cell (CIC)\xspace}
\newcommand{\PIC}{\renewcommand{\PIC}{PIC\xspace}Particle-In-Cell (PIC)\xspace}
\newcommand{\PM}{\renewcommand{\PM}{PM\xspace}Particle-Mesh (PM)\xspace}


\newcommand{\sump}[1]{\sum_{0{<}#1{<}\Delta X}}
\renewcommand{\wr}[1]{\mathop{W_{\mathrm{right}}}\nolimits(#1)}
\newcommand{\wl}[1]{\mathop{W_{\mathrm{left}}}\nolimits(#1)}

\newcommand{\mycomment}[1]{}

\renewcommand{\floatpagefraction}{.9}
\renewcommand{\textfraction}{.1}
\renewcommand{\topfraction}{.8}
\renewcommand{\bottomfraction}{.8}

%\def\tablesize{\footnotesize}
\def\tablesize{\small}

\bibliographystyle{plainnat}

\title{\picsim: a suite of 5D/6D Particle-In-Cell\\ plasma simulation codes\\
version~\version}
\author{Patrick Guio}
\date{\normalsize$ $Date: 2013/03/15 17:33:12 $ $,~ $ $Revision: 1.17 $ $}

\begin{document}

\maketitle

\tableofcontents

\section{Introduction}
\picsim is a suite of 5D (i.e.\ 2R+3V) and 6D (3R+3V) electrostatic collision
less plasma \PIC simulation codes \citep{dawson:1983,birdsall:1985}.  The
development of \picsim started in 2000, enhancement and new features are
added on a regular basis.

Each code solves the dynamic (in continuous 5D and 6D phase space) of charged
macro-particles in a Lagrangian frame subject to internal and external
forces, the internal electrostatic field is updated self-consistently on a
fixed Eulerian mesh This model that includes interactions of particles only
through the average fields are called Particle-Mesh (PM).  \picsim allows
external force on the particles driven by external electromagnetic, electric
and/or gravitation fields. The macro particles are organised in species
specified by their mass and charge as well as their density, mean velocity
and temperature and other control parameters for their temporal drive as well
as boundary conditions.

\begin{figure}[ht]
\begin{center}
\includegraphics[width=0.5\textwidth]{pic_cycle}
\end{center}
\caption[]{The basic cycle of a \PIC simulation.}
\label{fig:pic_cycle}
\end{figure}

\begin{figure}[ht]
\begin{center}
\includegraphics[width=0.5\textwidth]{pic_grid}
\end{center}
\caption[]{Sketch of the relation between particle charge $q$ and mesh charge 
density $\rho$.}
\label{fig:pic_grid}
\end{figure}

Fig.~\ref{fig:pic_cycle} and \ref{fig:pic_grid} sketches the basic principle
of the \PIC algorithm.
In Fig.~\ref{fig:pic_cycle} the basic cycle is illustrated and consists of four
operations performed at each iteration: (i) the particle weighting where
macro particle charge $q$ is deposited onto the charge density grid
$\rho_{ij}$/$\rho_{ijk}$ using a first order linear weighting scheme,
the \CIC scheme (Fig.~\ref{fig:pic_grid});
(ii) the field solver that provides the electrostatic potential $\Phi$
and electric field $\vec{E}$ on the mesh using the multigrid
method \citep{wesseling:1991}; (ii) the field weighting (reciprocal of the
particle weighting) where forces given on the grid are interpolated
to the position of the particles; and (iv) the particle mover that integrate
Newton second law equation using the leap frog method.

The code suite consists of several programs for different approximations and
are available for both 5D and 6D phase space.

Tables~\ref{table:scieinfo} and \ref{table:techinfo} presents a summary of
scientific information and technical details.

%Table with scientific information
\begin{table}[ht]
\caption{Scientific information about the code}
\label{table:scieinfo}
\vspace{2mm}
\begin{footnotesize}
%\begin{tabular}{|p{\firstcol}|p{\secondcol}|}
\begin{tabularx}{\textwidth}{|X|X|}
\hline
Class of phenomena that can be studied  &
a) Linear/nonlinear plasma wave-particle interactions, wave modes properties
propagation and saturation.
Low frequency electrostatic waves in magnetised/unmagnetised
and homogeneous/inhomogeneous plasmas subject to external forces,
and versatile configuration of species.\\
& b) Beam plasma interactions, phase space studies (vortices,
self-organised and solitary structures).\\
& c) Wave radiation pattern and wake structure from point charged (dust)
particle and finite size obstacle in plasma.\\
& c) Linear/nonlinear wave modes in dusty plasma.\\
\hline
%\multirow{Input parameters for the model}          & a) Object: 
Input parameters for the model          &
Note that all parameters have a default value and need not to be provided.
A list of all options together with information is provided by giving the
\texttt{--help} option to the command line. All options can be parsed
through a configuration file and can be redefined on the fly on the command
line\\
&
a) Plasma setup
\begin{itemize}
\item A list of names for the species to be included in the simulation
(several
species classes are available: background, beam, driven background/beam,
non homogeneous driven background).
\item For each species included, properties have to be set up
\begin{itemize}
\item Charge and mass
\item Density, drift velocity vector and independent temperature for
each direction.
\item Boundary conditions.
\item Geometry and spatial extent.
\item More\ldots
\end{itemize}
\end{itemize}
\\
 & b)  Simulation setup
\begin{itemize}
\item Simulation box dimension
\item Parameters for the multigrid solver (grid size, tolerance, boundary
conditions, max number of cycle, interpolation type, number of pre/post
smoothing)
\item Timestep and number of iterations.
\item Average number of simulation macro particles per unit volume cell
($\Lambda$ plasma parameter).
\item Boundary conditions for fields and particles.
\end{itemize}
\\
 & c)  Diagnostic setup
\begin{itemize}
\item Time specifications for the different diagnostics classes.
\item Enable/disable switch for most diagnostics.
\item Highly configurable setup for an arbitrary number of phase space
diagnostics.
\item Highly configurable setup for an arbitrary number of spectral
diagnostics.
\end{itemize}\\
\hline
Output of the model                     &
All output are stored in a HDF4 file using the Scientific Data Set (SDS)
which supports multi-dimensional gridded data with meta-data including
support
for an unlimited dimension.
\begin{itemize}
\item 2D/3D gridded data (density, potential) as function of time.
\item High resolution slice of gridded data.
\item Phase space diagnostics.
\item Moments diagnostics.
\item Spectral diagnostics ($S(k)$ and $S(k,\omega)$)
\item Log written including derived physical and numerical parameters,
simulation details, misc. diagnostics, etc.
\end{itemize}
\\
\hline
%\end{tabular}
\end{tabularx}
\end{footnotesize}
\end{table}

%Table with technical details
\begin{table}[ht]
\caption{Technical details about the code}
\label{table:techinfo}
\vspace{2mm}
\begin{footnotesize}
%\begin{tabular}{|p{\firstcol}|p{\secondcol}|}
\begin{tabularx}{\textwidth}{|X|X|}
\hline
Language the code is written in  & C++ \\
\hline
Basic requirements (memory, additional libraries, etc...)  & The code is
written in modern standard C++ and uses extensively the Standard Template
Library (STL). Configuration for different platforms is made easy by using
the GNU autotools
(\htmladdnormallink{autoconf}{http://www.gnu.org/software/autoconf},
\htmladdnormallink{automake}{http://www.gnu.org/software/automake/} and
\htmladdnormallink{libtool}{http://www.gnu.org/software/libtool/}).
It requires the following open source and freely available
libraries (available as package on most Linux distributions)
\begin{itemize}
\item \htmladdnormallink{Blitz++}{http://www.oonumerics.org/blitz}
\item \htmladdnormallink{HDF4}{http://www.hdfgroup.org/products/hdf4}
\item \htmladdnormallink{FFTW}{http://www.fftw.org} (version 2 or 3)
\end{itemize}
It is also MPI-enabled and tested with both
\htmladdnormallink{MPICH2}{http://www.mcs.anl.gov/mpi/mpich2} and
\htmladdnormallink{OpenMPI}{http://www.open-mpi.org}
\\
\hline
%\multirow{List of tested compilers/platforms} & gcc (4.6) x86-64 (Ubuntu
%11.10) \\
List of tested compilers/platforms & GNU g++ (4.x) on i386-linux \\
 & Intel icpc (12.x) on i386-linux, x86-64-linux and ia64-linux  \\
 & pathScale pathCC on i386-linux and x86-64-linux  \\
 & Portland pgCC on x86-64-linux  \\
 & IBM xlC (10.x) on SP6 \\
\hline
Team of developers/maintainers & Patrick Guio \\
\hline
Basic instructions/policy to use it &
\picsim is not open and not distributed outside of the team. Some details of
the code are published in several scientific articles in international
journals.
Simulation run can be performed on request. \\
\hline
References & \picsim simulations are published in nine papers
\begin{itemize}
\item Effect of a temperature gradient on low-frequency ion modes
\citep{guio:2001, guio:2010}
\item Beam particle interaction and organised structures
\citep{daldorff:2001,guio:2003a}
\item Waves and wake pattern caused by charged dust
\citep{guio:2003b,guio:2008a}
\item Waves, wake pattern and organised structures caused by finite size
obstacle \citep{guio:2004,guio:2005}
\end{itemize}\\
\hline
\end{tabularx}
%\end{tabular}
\end{footnotesize}
\end{table}


\section{Model equations}

\subsection{Electrostatic}

Let us consider for instance \picsim{i} which is an hybrid \PIC code that has
been widely used for publication where the electrons are represented as a
fluid with different approximation available (isothermally Boltzmann
distributed with temperature $\Te$ as well as deviation from that state)
while the ion populations are particles. In this case the self-consistent
electrostatic potential $\Phi$ is related to the species densities $n_j$ with
the following nonlinear Poisson equation.
\begin{falign}
\lapl[r]\Phi&=-\frac{e}{\epso}
\left(\sum_iZ_in_i-n_e\exp\frac{e\Phi}{\kB\Te}\right)\\
\vec{E}&=-\grad[r]{\Phi}
\end{falign}

The same set of equations are solved numerically in the particle mover,
i.e.\ for each particle of species $i$, the velocity
and position are given by
\begin{falign}
\diff{\vec{v}_i}{t}&=\frac{Z_ie}{m_i}
\left(\vec{E}+\vec{v}_i\cross\vec{B}\right)+\vec{G}\\
\diff{\vec{r}_i}{t}&=\vec{v}_i
\end{falign}
where \vec{B} is an externally imposed magnetic field and \vec{G} is
an externally imposed gravitational field.
Note that the internal gravity field \vec{g} of the system can
be added to \vec{G} by solving Gauss's law for gravity
\begin{falign}
\lapl[r]\varphi&=-4\pi G\left(\sum_in_i\right)\\
\vec{g}&=-\grad[r]{\varphi}
\end{falign}


\subsection{Dimensionless variables}

Let us define the mass $m_0$, density $n_0$ and the temperature $T_0$ and 
their dimensionless counterparts $\mu=m/m_0$, $\rho=n/n_0$ and
$\theta=T/T_0$. We can derive the corresponding thermal velocity $v_0$, 
plasma frequency $\omega_0$, Debye length $\lambda_0$ and plasma
parameter $\Lambda_0$

\begin{align*}
v_0&=\left(\frac{\kB T_0}{m_0}\right)^\frac{1}{2}, &
\omega_0&=\left(\frac{n_0e^2}{\epso m_0}\right)^\frac{1}{2}, &
\lambda_0&=\frac{v_0}{\omega_0}=
\left(\frac{\epso\kB T_0}{n_0e^2}\right)^\frac{1}{2}, &
\Lambda_0&=\lambda_0^d n_0
\end{align*}
where $d$ is the spatial dimension.

The following dimensionless variables are then defined
\begin{align*}
\vec{\xi}&=\frac{\vec{r}}{\lambda_0}, &
\vec{\psi}&=\frac{\vec{v}}{v_0}, &
\tau&=\omega_0 t,
\end{align*}
and the relation between the differential operators are
\begin{align*}
\diff{\vec{r}}{t}&=\lambda_0\omega_0\diff{\vec{\xi}}{\tau}=
v_0\diff{\vec{\xi}}{\tau}, &
\grad[r]&=\frac{1}{\lambda_0}\grad[\xi], & 
\lapl[r]&=\frac{1}{\lambda_0^2}\lapl[\xi], &
\diff{v}{t}&=v_0\omega_0\diff{\vec{\psi}}{\tau}
\end{align*}

The model equations can be rewritten 
\begin{align}
\frac{e}{\kB T_0}\lapl[\xi]\Phi&=
-\sum_iZ_i\rho_i+\rho_e\exp\frac{e\Phi}{\kB\Te}\\
\vec{E}&=-\frac{1}{\lambda_0}\grad[\xi]\Phi 
\end{align}
and
\begin{align}
v_0\omega_0\diff{\vec{\psi}_i}{\tau}&=\frac{eZ_i}{m_i}
\left(\vec{E}+v_0\vec{\psi}_i\cross\vec{B}\right)+\vec{G}\\
\lambda_0\omega_0\diff{\vec{\xi}_i}{\tau}&=v_0\vec{\psi}_i
\end{align}

Posing
\begin{align*}
\Phi_0&=\frac{\kB T_0}{e},&
\phi&=\frac{\Phi}{\Phi_0} &\\
E_0&=\frac{\Phi_0}{\lambda_0},&
\vec{\epsilon}&=\frac{\vec{E}}{E_0} &\\
B_0&=\left(\frac{n_0m_0}{\epso}\right)^\frac{1}{2},&
\vec{\Omega_0}&=\frac{e}{m_0}\vec{B},&
\vec{\beta}&=\frac{\vec{B}}{B_0}=\frac{\vec{\Omega_0}}{\omega_0} &\\
G_0&=v_0\omega_0, &
\vec{\zeta}&=\frac{\vec{G}}{G_0}
\end{align*}

leads to the following normalised model equations
\begin{falign}
\lapl[\xi]\phi&=-\sum_iZ_i\rho_i+\rho_e\exp\frac{\phi}{\theta_e}\\
\vec{\epsilon}&=-\grad[\xi]{\phi}\\
\diff{\vec{\psi}_i}{\tau}&=\frac{Z_i}{\mu_i}
\left(\vec{\epsilon}+\vec{\psi}_i\cross\vec{\beta}\right)+
\vec{\zeta}\\
\diff{\vec{\xi}_i}{\tau}&=\vec{\psi}_i
\end{falign}

\subsection{Electromagnetic}

Let us consider Gauss' law and Ampère's law from Maxwell equations
which couples the electromagnetic observables \vec{E} and \vec{B} to the 
source terms \vec{J} and $\rho$, formally
\begin{align}
\divg[r]\vec{E}&= \frac{\rho}{\epso},\label{eq:gauss1}\\
\curl[r]\vec{B}&=\muo\vec{J}+\muo\epso\pdiff{t}\vec{E}\label{eq:ampere}
\end{align}

These equations hide a dependency between the densities $\rho$ and \vec{J}. 
By evaluating the divergence of Ampère's law \Eq{\ref{eq:ampere}} and adding
the time derivative of Gauss law \Eq{\ref{eq:gauss1}}, we obtain
an equation for conservation of charge
\begin{align}
\pdiff{t}\rho + \divg[r]\vec{J} = 0,
\end{align}
which means that the change of the density $\rho$ is caused by the
sources of the current density \vec{J}.

Next, let us consider Gauss' second law from Maxwell equations, 
formally
\begin{align}
\divg[r]\vec{B}=0.\label{gauss2}
\end{align}
The mathematical structure of this equation means that \vec{B} can be derived
from a vector potential \vec{A}, formally
\begin{align}
\vec{B}=\curl[r]{A}.\label{eq:rot}
\end{align}

Finally, let us consider Faraday's law from Maxwell equations, formally
\begin{align}
\curl[r]{\vec{E}}=-\pdiff{t}\vec{B},\label{eq:faraday}
\end{align}
and substitute \vec{B} by the vector potential definition, formally
\begin{align}
\curl[r]{(\vec{E} + \pdiff{t}\vec{A})}=\vec{0}.
\end{align}
This mathematical
structure allow to represent $\vec{E} + \pdiff{t}\vec{A}$ as some gradient,
formally
\begin{align}
\vec{E} + \pdiff{t}\vec{A} = -\grad[r]{\Phi}.\label{eq:grad}
\end{align}

Thus, both \vec{E} and \vec{B} can be represented by means of some potentials
\vec{A} and $\Phi$ as follows
\begin{align}
\vec{E}&=-(\grad[r]\Phi+\pdiff{t}\vec{A}),\label{eq:elecpot}\\
\vec{B}&=\curl[r]\vec{A}.\label{eq:magpot}
\end{align}

\subsubsection{Gauge transformation}
Introducing an arbitrary function $\psi$, the identity
\begin{align}
\grad[r]{\Phi}+\pdiff{t}\vec{A}=
\grad[r]{(\Phi-\pdiff{t}\psi)}+\pdiff{t}(\vec{A}+\grad[r]{\psi})
\end{align}
shows that the substitutions
\begin{align}
\Phi\quad\rightarrow\quad
\Phi^\prime  & = \Phi-\pdiff{t}\psi\label{eq:subst1}\\
\vec{A}\quad\rightarrow\quad
\vec{A}^\prime & = \vec{A} + \grad[r]{\psi},\label{eq:subst2}
\end{align}
generates equivalent potentials $\Phi^\prime$ and $\vec{A}^\prime$ as 
$\Phi$ and $\vec{A}$ for the representation of the observables
\vec{E}, \vec{B} for any arbitrary function $\psi$. 
Also the observables \vec{J} and $\rho$  given by
\Eqs{\ref{eq:gauss1}--\ref{eq:ampere}} remain unchanged under the
transformations \Eqs{\ref{eq:subst1}--\ref{eq:subst2}}.

The transformations \Eqs{\ref{eq:subst1}--\ref{eq:subst2}} leave 
the observables \vec{E}, \vec{B}, \vec{J} and 
$\rho$ invariant for any arbitrary functions $\psi$. 
The substitutions, as defined by \Eqs{\ref{eq:subst1}--\ref{eq:subst2}},
are called a gauge transformation with the generating function $\psi$.

Let us now assume that the source densities $\rho$ and \vec{J} are given. 
We still have to show the existence of potentials \vec{A}  and $\Phi$, 
which fulfil the Maxwell equations \Eqs{\ref{eq:gauss1}--\ref{eq:ampere}}.
Inserting the potential representations \Eqs{\ref{eq:elecpot}--\ref{eq:magpot}}
into Ampère's law \Eq{\ref{eq:ampere}} yields 
\begin{align}
\curl[r]{\left(\curl[r]{\vec{A}}\right)} + 
\muo\epso\left(\pdiff[2]{t}\vec{A}+\grad[r]{\left(\pdiff{t}\Phi\right)}\right)
= \muo\vec{J}.
\end{align}
We make use of the identity 
$\curl\left(\curl\vec{A}\right)=\grad\left(\divg\vec{A}\right)-\lapl\vec{A}$
and obtain
\begin{align}
\muo\epso\pdiff[2]{t}\vec{A} - \lapl[r]\vec{A} + 
\grad[r]\left(\divg[r]\vec{A} + \muo\epso\pdiff{t}\Phi\right) = 
\muo\vec{J}.\label{eq:amperealt}
\end{align}

Analogously inserting the representations 
\Eqs{\ref{eq:elecpot}--\ref{eq:magpot}} into Gauss' law \Eq{\ref{eq:gauss1}},
we obtain
\begin{align}
-\left(\lapl[r]\Phi + \divg[r]\left(\pdiff{t}\vec{A}\right)\right)
= \frac{\rho}{\epso}, 
\end{align}
or 
\begin{align}
\muo\epso\pdiff[2]{t}\Phi - \lapl[r]\Phi - 
\pdiff{t}\left(\divg[r]\vec{A} + \muo\epso\pdiff{t}\Phi\right)
=  \frac{\rho}{\epso}.\label{eq:gauss1alt}
\end{align}

It is then easy to check that the application of the gauge transformation
\Eqs{\ref{eq:subst1}--\ref{eq:subst2}} transforms 
\Eq{\ref{eq:amperealt}} and \Eq{\ref{eq:gauss1alt}}  into
\begin{align}
\muo\epso\pdiff[2]{t}\vec{A}^\prime - \lapl[r]\vec{A}^\prime + 
\grad[r]\left(\divg[r]\vec{A}^\prime + \muo\epso\pdiff{t}\Phi^\prime\right) & = 
\muo\vec{J},\label{eq:amperealt1}\\
\muo\epso\pdiff[2]{t}\Phi^\prime - \lapl[r]\Phi^\prime - 
\pdiff{t}\left(\divg[r]\vec{A}^\prime + \muo\epso\pdiff{t}\Phi^\prime\right)
& =  \frac{\rho}{\epso},\label{eq:gauss1alt1}
\end{align}
i.e.\ both equations are gauge invariant.

At the present state, it is not possible to  guarantee the existence of
solutions \vec{A} and $\Phi$ of these coupled differential equations for given
source functions $\rho$ and \vec{J}. 
However, using the freedom of gauge transformations it
is possible to show that the equations can be transformed in a decoupled
version, which is solvable. 
This implies that \Eq{\ref{eq:amperealt1}} and \Eq{\ref{eq:gauss1alt1}} 
are solvable too.

\subsubsection{Coulomb gauge}
The problem in solving \Eq{\ref{eq:amperealt}} and \Eq{\ref{eq:gauss1alt}}
is their coupling. One way of decoupling them is to introduce the additional
condition, called the Coulomb (or transverse) gauge formulation)
\begin{align}
\divg[r]\vec{A}=0, \label{eq:coulombgauge}
\end{align}
i.e.\ we have to remove the term $\divg[r]\vec{A}$ in
\Eq{\ref{eq:gauss1alt}} by applying
an appropriate gauge transformation. Since
\Eqs{\ref{eq:subst1}--\ref{eq:subst2}} transform
\begin{align}
\divg[r]\vec{A} \quad \rightarrow \quad
\divg[r]\vec{A}^\prime = \divg[r]\vec{A} + \lapl[r]\psi
\end{align}
we merely have to choose the generating function $\psi$ as a solution of 
\begin{align}
\lapl[r]\Psi = - \divg[r]\vec{A}.\label{eq:psicond}
\end{align}
This gauge transformation transforms \Eq{\ref{eq:gauss1alt}} into 
\begin{align}
\muo\epso\pdiff[2]{t}\Phi^\prime - \lapl[r]\Phi^\prime -
\muo\epso\pdiff[2]{t}\Phi^\prime  = \frac{\rho}{\epso}, 
\end{align}
or simply
\begin{align}
\lapl[r]\Phi^\prime = -\frac{\rho}{\epso}, 
\end{align}
which is the Poisson equation with the well-known integral solution
\begin{align}
\Phi^\prime(\vec{x},t) = \frac{1}{4\pi\epso}
\int\frac{\rho(\vec{y},t)}{|\vec{x}-\vec{y}|} d\vec{y}. 
\end{align}

Therefore $\Phi^\prime$ is now a known function for the other equation, 
which can be rewritten as
\begin{align}
\muo\epso\pdiff[2]{t}\vec{A}^\prime - \divg[r]\vec{A}^\prime =
\muo\vec{J}^\ast, 
\end{align}
where $\vec{J}^\ast = \vec{J}-\epso\grad[r](\pdiff{t}\Phi)$ is a known function.
Again a solution of this equation in integral form is known
\begin{align}
\vec{A}(\vec{x},t) = \frac{\muo}{4\pi} 
\int\frac{\vec{J}^\ast(\vec{y},t-|\vec{x}-\vec{y}|/c)}{|\vec{x}-\vec{y}|}
d\vec{y}.
\end{align}

However, something is left. One has to show that the solution
\Eq{\ref{eq:psicond}} fulfils the condition \Eq{\ref{eq:coulombgauge}}, i.e.\
that $\divg\vec{A} = 0$, which is a rather technical proof and is left out.

\subsubsection{Lorenz gauge}
The second method of decoupling \Eq{\ref{eq:amperealt}} and
\Eq{\ref{eq:gauss1alt}} is more symmetrical. Let us introduce the so-called
Lorenz convention
\begin{align}
\divg[r]\vec{A}+\muo\epso\pdiff{t}\Phi = 0.\label{eq:lorenzconv}
\end{align}

This goal can be attained by an appropriate gauge transformation. Under the
gauge transformations \Eqs{\ref{eq:subst1}--\ref{eq:subst2}} we have
\begin{align}
\divg[r]\vec{A}+\muo\epso\pdiff{t}\Phi \quad\rightarrow\quad
\divg[r]\vec{A}+\muo\epso\pdiff{t}\Phi+
\left(\lapl[r]\Psi - \muo\epso \pdiff[2]{t}\Psi\right)
\end{align}
Thus if we take the generating function $\Psi^{\prime\prime}$ to be a solution
of the inhomogeneous wave equation
\begin{align}
\muo\epso\pdiff[2]{t}\Psi^{\prime\prime}-\lapl[r]\Psi^{\prime\prime}
= \divg[r]\vec{A}+\muo\epso\pdiff{t}\Phi
\end{align}
we obtain equivalent potentials $\vec{A}^{\prime\prime}$ and
$\Phi^{\prime\prime}$ that fulfil
\begin{align}
\muo\epso\pdiff[2]{t}\vec{A}^{\prime\prime}-\lapl[r]\vec{A}^{\prime\prime} 
&= \muo\vec{J},\\
\muo\epso\pdiff[2]{t}\Phi^{\prime\prime}-\lapl[r]\Phi^{\prime\prime}
&= \frac{\rho}{\epso}.
\end{align}
And both equations have solutions in integral form
\begin{align}
\vec{A}^{\prime\prime} &= \frac{\muo}{4\pi} 
\int\frac{\vec{J}(\vec{y},t-|\vec{x}-\vec{y}|/c)}{|\vec{x}-\vec{y}|},
\label{eq:Alorenzsol}\\
\Phi^{\prime\prime} &= \frac{1}{4\pi\epso}
\int\frac{\rho(\vec{y},t)}{|\vec{x}-\vec{y}|} d\vec{y}.
\label{eq:philorenzsol}
\end{align}
And again the question arises whether the solutions 
\Eqs{\ref{eq:Alorenzsol}--\ref{eq:philorenzsol}} fulfil the Lorenz
convention \Eq{\ref{eq:lorenzconv}}. 

\subsubsection{Gauge equivalence}
On first look it seems doubtful whether the Coulomb gauge and the Lorenz
gauge would yield the same observables \vec{E} and \vec{B} for identical
sources \vec{J} and $\rho$. However it is easy to show that both solutions 
$\vec{A}^\prime, \Phi^\prime$ and $\vec{A}^{\prime\prime}, \Phi^{\prime\prime}$
are transformable into each other by a gauge transformation.

Let $(\vec{E},\vec{B})(\vec{A},\Phi)$ denote the observables \vec{E} and
\vec{B} derived from the potentials \vec{A} and $\Phi$ by \Eq{\ref{eq:rot}}
and \Eq{\ref{eq:grad}}. Then   due to the gauge equivalence between 
$\vec{A},\Phi$ and $\vec{A}^\prime,\Phi^\prime$ we have
\begin{align}
(\vec{E},\vec{B})(\vec{A}^\prime,\Phi^\prime)=(\vec{E},\vec{B})(\vec{A},\Phi)
\end{align}
and due to the gauge equivalence between $\vec{A},\Phi$ and
$\vec{A}^{\prime\prime},\Phi^{\prime\prime}$ 
\begin{align}
(\vec{E},\vec{B})(\vec{A}^{\prime\prime},\Phi^{\prime\prime})=
(\vec{E},\vec{B})(\vec{A},\Phi)
\end{align}
and therefore
\begin{align}
(\vec{E},\vec{B})(\vec{A}^\prime,\Phi^\prime)=
(\vec{E},\vec{B})(\vec{A}^{\prime\prime},\Phi^{\prime\prime})
\end{align}
This result is a special case of the general result on gauge invariance of
the observables that we developed in the previous sections.

The possibility of representing the observables of the Maxwell theory by
different  (gauge-equivalent) potentials means that within the Maxwell
theory the electrodynamic potential \vec{A} and $\Phi$ possess no physical
reality. 


\subsubsection{Old stuff}
\begin{falign}
\lapl[r]\Phi+\pdiff{t}\left(\divg[r]\vec{A}\right)&=
-\frac{e}{\epso}\left(\sum_iZ_in_i-n_e\exp\frac{e\Phi}{\kB\Te}
\right)\label{eq:gauss}\\
\lapl[r]\vec{A}-\muo\epso\pdiff[2]{t}\vec{A}&=-\muo\vec{J}+
\grad[r]\left(\divg[r]\vec{A}+
\muo\epso\grad[r]\pdiff{t}\Phi\right)\\
\vec{E}&=-\grad[r]\Phi-\pdiff{t}\vec{A}\\
\vec{B}&=\curl[r]\vec{A}
\end{falign}
The Coulomb (or transverse) gauge formulation is used. In this gauge,
$\divg[r]\vec{A}=0$.
\begin{falign}
\lapl[r]\Phi&=-\frac{e}{\epso}
\left(\sum_iZ_in_i-n_e\exp\frac{e\Phi}{\kB\Te}\right)\\
\lapl[r]\vec{A}-\muo\epso\pdiff[2]{t}\vec{A}&=-\muo\vec{J}+
\muo\epso\grad[r]\pdiff{t}\Phi
\end{falign}
and taking the gradient of Ampère's law leads to
\begin{falign}
\lapl[r]\Phi&=-\frac{e}{\epso}
\left(\sum_iZ_in_i-n_e\exp\frac{e\Phi}{\kB\Te}\right)\\
\lapl[r]\pdiff{t}\Phi&=\frac{1}{\epso}\divg[r]\vec{J}
\end{falign}

\section{Density calculation}

Let us assume a $n$-points regular grid $G_n=\{X_1,\ldots, X_n\}$ with 
space interval $\Delta X = (X_n-X_1)/(n-1)$ on the domain $\Omega=[X_1, X_n]$. 

We want to estimate the density $\rho(G_n)$ of a set of particles $P$ 
where particle $P_j$ has coordinate $x_j$.

The density $\rho_i(X_i)$ at grid point $X_i$ not on the boundary 
can be written
\begin{falign}
\rho(X_i)=\sump{x_j{-}X_i}\wl{x_j{-}X_i}+\sump{X_i{-}x_j}\wr{X_i{-}x_j}
\end{falign}

For a first-order weighting (Cloud-In-Cell method), the two weights 
functions $W_{\mathrm{left}}$  and $W_{\mathrm{right}}$ are defined as
\begin{align*}
\wl{x}&=\frac{x}{\Delta X},& \wr{x}&=1-\frac{x}{\Delta X}
\end{align*}

\subsection{Density at boundary points for periodic condition}
For periodic boundary condition over the domain $\Omega=[X_1, X_n]$, we have
\begin{equation}
\rho(x{+}X_n{-}X_1)=\rho(x), \qquad\forall x
\end{equation}

Applying the periodic condition at the boundary points leads to
\begin{falign}
\rho(X_1)=\rho(X_n)=\sump{x_j{-}X_n}\wl{x_j{-}X_n}+
\sump{X_1{-}x_j}\wr{X_1{-}x_j}
\end{falign}

\subsection{Density at boundary points for Dirichlet condition}
When the potential $\phi=0$ at the boundary, we call  
the finite screen theorem and says that in order to keep the potential to
zero the charge distribution on one side of the boundary needs to be the 
opposite of the charge distribution of the other side, therefore
\begin{falign}
\rho(X_1)&=-\rho_e\\
\rho(X_n)&=-\rho_e
\end{falign}

\subsection{Density at boundary points for Neumann condition}
When the potential at the boundary is such that
\begin{align}
\diffp{\phi}{x}(X_1)&=f(X_1)\\
\diffp{\phi}{x}(X_n)&=f(X_n)
\end{align}

With the Boltzmann approximation
\begin{equation}
\frac{n}{n_0}\sim\frac{e\phi}{\Te}
\end{equation}
One can write
\begin{align}
\dint{n}=\frac{en_0}{\Te}\diffp{\phi}{x}\dint{x}
\qquad\Rightarrow\qquad
n(\Delta X)=n_0(1+\frac{e}{\Te}\diffp{\phi}{x}\Delta X)
\end{align}


\begin{falign}
\rho_{\mathrm{left}}(X_1)&=\sump{x_j{-}X_1}\wr{x_j{-}X_1}\left(2-
\frac{e}{\Te}f(X_1)\Delta X\right)\\
\rho_{\mathrm{right}}(X_n)&=\sump{X_n{-}x_j}\wl{X_n{-}x_j}\left(2+
\frac{e}{\Te}f(X_n)\Delta X\right)
\end{falign}

In the case of a gravitation field $G$, the equilibrium is defined by 
\begin{equation}
\phi=\frac{MG}{e}z \qquad\Rightarrow\qquad\diffp{\phi}{x}(X_1)=
\diffp{\phi}{x}(X_n)=\frac{MG}{e}
\end{equation}
and
\begin{falign}
\rho_{\mathrm{left}}(X_1)&=\sump{x_j{-}X_1}\wr{x_j{-}X_1}\left(2-
\frac{MG}{\Te}\Delta X\right)\\
\rho_{\mathrm{right}}(X_n)&=\sump{X_n{-}x_j}\wl{X_n{-}x_j}\left(2+
\frac{MG}{\Te}\Delta X\right)
\end{falign}

\section{Energy}

\subsection{Energy equation}
The 1-d Vlasov equation for the ion species $i$ with distribution function
$f_i$ is multiplied by $1/2m_iv^2$
and integrated over the phase space $(x,v)$ to give the energy equation
\begin{equation}
\diffp{}{t}\iint\frac{1}{2}m_iv^2f_i\dint{x}\dint{v}+
\int\frac{1}{2}m_iv^3\left[\int\diffp{f_i}{x}\dint{x}\right]
\dint{v}-\iint\frac{1}{2}Z_iev^2\diffp{\Phi}{x}
\diffp{f_i}{v}\dint{x}\dint{v}=0
\end{equation}
For periodic and insulated systems
\begin{equation}
\int\diffp{f_i}{x}\dint{x}=0,\qquad\mbox{i.e.}\qquad
\left[f_i(x,v,t)\right]_{\Gamma_x}=0
\end{equation}
The last term is $v$-integrated (assuming that 
$[v^2 f_i]_{\Gamma_v}=0$)
then $x$-integrated (assuming that $[\Phi u]_{\Gamma_x}=0$)
by part and the energy equation is written
\begin{align}
\diffp{}{t}\iint\frac{1}{2}m_iv^2f_i\dint{x}\dint{v}+
\frac{1}{2}\iint Z_ie2v\diffp{\Phi}{x}f_i\dint{x}\dint{v}&=0\\
\diffp{}{t}\iint\frac{1}{2}m_iv^2f_i\dint{x}\dint{v}-Z_ie
\int\Phi\diffp{}{x}\left[\int vf_i\dint{v}\right]\dint{x}&=0
\end{align}
The continuity equation for species $i$ (Vlasov equation integrated over
$v$) is written
\begin{equation}
\diffp{}{t}\int f_i\dint{v}+\diffp{}{x}\int vf_i\dint{v}=0
\end{equation}
Therefore the energy equation is rewritten
\begin{equation}
\diffp{}{t}\iint\frac{1}{2}m_iv^2f_i\dint{x}\dint{v}+Z_ie
\int\Phi\diffp{}{t}\left[\int f_i\dint{v}\right]\dint{x}=0
\end{equation}
The Poisson equation is written
\begin{equation}
\diffp[2]{\Phi}{x}=-\frac{e}{\epso}
\left(\sum_iZ_i\int f_i\dint{v}-n_e\exp\left[\frac{e\Phi}{\kB\Te}\right]\right)
\end{equation}
Adding the Vlasov for each species and identifying the terms from Poisson
equation gives
\begin{equation}
\diffp{}{t}\sum_i\iint\frac{1}{2}m_iv^2f_i\dint{x}\dint{v}-
\int\Phi\diffp{}{t}\left[\epso\diffp[2]{\Phi}{x}\right]\dint{x}+
\int\Phi\diffp{}{t}\left[en_e\exp\left(\frac{e\Phi}{\kB\Te}\right)\right]
\dint{x}=0
\end{equation}
Applying product time derivation rule on the second term gives
\begin{align}
\diffp{}{t}
\sum_i\iint\frac{1}{2}m_iv^2f\dint{x}\dint{v}&-
\diffp{}{t}\int\Phi\epso\diffp[2]{\Phi}{x}\dint{x}+
\int\diffp{\Phi}{t}\epso\diffp[2]{\Phi}{x}\dint{x}+\nonumber\\
&\int\Phi\diffp{}{t}\left[en_e\exp\left(\frac{e\Phi}{\kB\Te}\right)\right]
\dint{x}=0\\
\end{align}
Integrating by part the second and third terms (assuming 
$\left[\Phi\diffp{\Phi}{x}\right]_{\Gamma_x}=0$ and 
$\left[\diffp{\Phi}{t}\diffp{\Phi}{x}\right]_{\Gamma_x}=0$) 
and applying product time derivation rule on the fourth term
\begin{align}
\diffp{}{t}\sum_i\iint\frac{1}{2}m_iv^2f\dint{x}\dint{v}&+
\diffp{}{t}\int\epso\left(\diffp{\Phi}{x}\right)^2\dint{x}-
\int\diffp{\Phi}{xt}\epso\diffp{\Phi}{x}\dint{x}-\nonumber\\
&\diffp{}{t}\int\left(\kB\Te-e\Phi\right)
n_e\exp\left(\frac{e\Phi}{\kB\Te}\right)\dint{x}=0
\end{align}
Since 
\begin{equation}
\diffp{\Phi}{xt}\diffp{\Phi}{x}=\frac{1}{2}\left(\diffp{\Phi}{x}\right)^2
\end{equation}

The energy equation is written
\begin{falign}
\diffp{}{t}\left[\sum_i\iint\frac{1}{2}m_iv^2f\dint{x}\dint{v}+\right.&
\frac{\epso}{2}\int\left(\diffp{\Phi}{x}\right)^2\dint{x}-\nonumber\\
&\left.\int\left(\kB\Te-e\Phi\right)n_e\exp\left(\frac{e\Phi}{\kB\Te}\right)
\dint{x}\right]=0
\end{falign}

\subsection{Kinetic energy}
The density $n$, drift velocity $\vec{u}$ and temperature $T$ are defined as 
\begin{align}
n(\vec{r})&=\int f(\vec{r},\vec{v})\dint{\vec{v}}\\
\vec{u}(\vec{r})&=\int\vec{v} f(\vec{r},\vec{v})\dint{\vec{v}}\\
\frac{d}{2}n(\vec{r})\kB T(\vec{r})&=\frac{1}{2}m\int(\vec{v}-\vec{u})^2
f(\vec{r},\vec{v})\dint{\vec{v}}\\
&=\frac{1}{2}m\int\vec{v}^2 f(\vec{r},\vec{v})\dint{\vec{v}}
-\frac{1}{2}mn(\vec{r})u^2(\vec{r}) 
\end{align}
where $d=\mathrm{card}(\vec{v})$.

The mean kinetic energy $\mathrm{KE}$ is then defined as
\begin{equation}
\mathrm{KE}=\frac{\displaystyle\int\left(n(\vec{r})\kB T(\vec{r})
+\frac{1}{d}mn(\vec{r})u(\vec{r})\right)\dint{\vec{r}}}
{\displaystyle\int n(\vec{r})\dint{\vec{r}}}
\end{equation}

For Boltzmann distributed electron, the description is fluid and the 
three moments $n$, $\vec{u}$ and $\Te$ are given 
\begin{align}
\vec{u}&=\vec{0}\\
n(\vec{r})T(\vec{r})&=n_e
\exp\left[\frac{e\Phi(\vec{r})}{\kB\Te(\vec{r})}\right]\Te(\vec{r})
\end{align}
and
\begin{falign}
\mathrm{KE}_e=\kB T_0\frac{\displaystyle
\int\theta_e\exp\left({\phi/\theta_e}\right)\dint{\vec{\xi}}}
{\displaystyle\int\exp\left({\phi/\theta_e}\right)\dint{\vec{\xi}}}
\end{falign}

The ions consists of $N_p$ particles therefore the density distribution
is represented as 
\begin{equation}
f(\vec{r},\vec{v})= \sum_{i=1}^{N_p}
\dirac{\vec{r}-\vec{R}_i}\dirac{\vec{v}-\vec{V}_i}
\end{equation}
which gives
\begin{align}
n&=\sum_{i=1}^{N_p}\dirac{\vec{r}-\vec{R}_i}\\
nT+\frac{1}{d}mnu^2&=&\frac{1}{d}m\sum_{i=1}^{N_p}\vec{v}_i^2
\dirac{\vec{r}-\vec{R}_i}
\end{align}
and
\begin{falign}
\mathrm{KE}_i=\kB T_0\frac{\mu}{d}\frac{1}{N_p}\sum_{i=1}^{N_p}\vec{\psi}_i^2 
\end{falign}

\subsection{Field energy}
The electric field energy $\mathrm{ESE}$ is defined as
\begin{equation}
\mathrm{ESE}=\frac{1}{2}\epso\int\vec{E}^2\dint{\vec{r}}
\end{equation}
After an integration by part 
\begin{falign}
\mathrm{ESE}=\frac{\kB T_0}{2}\int\phi
\left[\rho_e\exp\left(\frac{\phi}{\theta_e}\right)-\sum_iZ_i\rho_i\right]
\dint{\vec{\xi}}
\end{falign}

\section{Random number generator}

\subsection{Uniform $U(a,b)$}
\begin{align}
f(x)&=\frac{1}{b-a}\\
\mu&=\frac{b+a}{2}\\
\sigma^2&=\frac{(b-a)^2}{12}
\end{align}

\subsection{Normal $N(\mu,\sigma)$}
\begin{equation}
f(x)=\frac{1}{\sqrt{2\pi}\sigma}\exp\left(-\frac{(x-\mu)^2}{2\sigma^2}\right)
\end{equation}

\subsection{Exponential $E(\lambda)$}
\begin{align}
f(x)&=\frac{1}{\lambda}\exp\left(-\frac{x}{\lambda}\right)\\
\mu&=\lambda\\
\sigma^2&=\lambda^2
\end{align}

\subsection{Rayleigh $R(s)$}
\begin{align}
f(x)&=\frac{x}{s^2}\exp\left(-\frac{x^2}{2s^2}\right)\\
\mu&=\sqrt{\frac{\pi}{2}}s\\
\sigma^2&=2s^2-\mu^2=2\left(1-\frac{\pi}{4}\right)s^2
\end{align}

\subsection{Flux $F(m,s)$}
\begin{align}
f(x)&=\frac{1}{\beta}\frac{x}{\sqrt{2\pi}s}
\exp\left(-\frac{(x-m)^2}{2s^2}\right)\\
\beta&=\frac{1}{\sqrt{2\pi}}s\exp\left[-\frac{m^2}{2s^2}\right]+
\frac{m}{2}\left[1+\erf\left[\frac{m}{\sqrt{2}s}\right]\right]\\
\mu&=\frac{1}{\beta}\left[\frac{m^2+s^2}{2}\left[1+
\erf\left[\frac{m}{\sqrt{2}s}\right]\right]+
\frac{ms}{\sqrt{2\pi}}\exp\left[-\frac{m^2}{2s^2}\right]\right]\\
\sigma^2&=\frac{1}{\beta}\left[m\frac{m^2+3s^2}{2}
\left[1+\erf\left[\frac{m}{\sqrt{2}s}\right]\right]+
s\frac{m^2+2s^2}{\sqrt{2\pi}}\exp\left[-\frac{m^2}{2s^2}\right]\right]-\mu^2
\end{align}

\subsection{Perturbation $P(m,s,\alpha)$}
\subsubsection*{Condition $|\alpha|<\sqrt{\frac{\pi}{2}}s\left[
\erf\left[\frac{1-m}{\sqrt{2}s}\right]+
\erf\left[\frac{m}{\sqrt{2}s}\right]\right]$}
\begin{align}
f(x)&=\frac{1}{\beta}\left[1+\frac{k}{\sqrt{2\pi}s}
\exp\left[-\frac{(x-m)^2}{2s^2}\right]\right]\\
k&=2\alpha\left[\erf\left[\frac{1-m}{\sqrt{2}s}\right]+
\erf\left[\frac{m}{\sqrt{2}s}\right]\right]^{-1}\\
\beta&=1+\frac{k}{2}\left[\erf\left[\frac{1-m}{\sqrt{2}s}\right]+
\erf\left[\frac{m}{\sqrt{2}s}\right]\right]\\
\mu&=\frac{1}{\beta}\left[\frac{1}{2}-\frac{ks}{\sqrt{2\pi}}\left[
\exp\left[-\frac{(1-m)^2}{2s^2}\right]-\exp\left[-\frac{m^2}{2s^2}\right]
\right]+\right.\nonumber\\
&\left.\frac{km}{2}\left[\erf\left[\frac{1-m}{\sqrt{2}s}\right]+
\erf\left[\frac{m}{\sqrt{2}s}\right]\right]\right]\\
\sigma^2&=\frac{1}{\beta}\left[\frac{1}{3}-
\frac{ks}{\sqrt{2\pi}}\left[(1+m)\exp\left[-\frac{(1-m)^2}{2s^2}\right]-
m\exp\left[-\frac{m^2}{2s^2}\right]\right]+\right.\nonumber\\
&\left.k\frac{m^2+s^2}{2}\left[\erf\left[\frac{1-m}{\sqrt{2}s}\right]+
\erf\left[\frac{m}{\sqrt{2}s}\right]\right]\right]-\mu^2
\end{align}

\subsubsection*{Bivariate perturbation $P(m_x,s_x,m_y,s_y,\alpha)$}
\begin{align}
f(x,y)&=\frac{1}{\beta_{xy}}\left[1+\frac{k^2}{{2\pi}s_xs_y}
\exp\left[-\frac{(x-m_x)^2}{2s_x^2}-\frac{(y-m_y)^2}{2s_y^2}\right]\right]\\
k^2&=4\alpha\left[\erf\left[\frac{1-m_x}{\sqrt{2}s_x}\right]+
\erf\left[\frac{m_x}{\sqrt{2}s_x}\right]\right]^{-1}
\left[\erf\left[\frac{1-m_y}{\sqrt{2}s_y}\right]+
\erf\left[\frac{m_y}{\sqrt{2}s_y}\right]\right]^{-1}\\
\beta_{xy}&=1+\frac{k^2}{4}\left[\erf\left[\frac{1-m_x}{\sqrt{2}s_x}\right]+
\erf\left[\frac{m_x}{\sqrt{2}s_x}\right]\right]
\left[\erf\left[\frac{1-m_y}{\sqrt{2}s_y}\right]+
\erf\left[\frac{m_y}{\sqrt{2}s_y}\right]\right]
\end{align}

Applying Bayes theorem $f(x,y)=f(x)f(y|x)$ leads to

\begin{align}
f(y|x)&=\frac{1}{\beta_{y|x}}\left[1+\frac{k}{\sqrt{2\pi}s_x}
\exp\left[-\frac{(x-m_x)^2}{2s_x^2}\right]\frac{k}{\sqrt{2\pi}s_y}
\exp\left[-\frac{(y-m_y)^2}{2s_y^2}\right]\right]\\
\alpha_{y|x}&=\frac{1}{\sqrt{2\pi}s_x}
\exp\left[-\frac{(x-m_x)^2}{2s_x^2}\right]
2\alpha\left[\erf\left[\frac{1-m_x}{\sqrt{2}s_x}\right]+
\erf\left[\frac{m_x}{\sqrt{2}s_x}\right]\right]^{-1}\\
\beta_{y|x}&=1+\frac{k^2}{2}\frac{1}{\sqrt{2\pi}s_x}
\exp\left[-\frac{(x-m_x)^2}{2s_x^2}\right]
\left[\erf\left[\frac{1-m_x}{\sqrt{2}s_x}\right]+
\erf\left[\frac{m_x}{\sqrt{2}s_x}\right]\right]
\end{align}


%\subsubsection*{$\alpha<0$}
%\begin{align}
%f(x)=&\frac{1}{\beta}\left[1-\frac{k}{\sqrt{2\pi}s}
%\left[1-\exp\left[-\frac{(x-m)^2}{2s^2}\right]\right]\right]\\
%k=&2\alpha\left[-\frac{2\alpha}{\sqrt{2\pi}s}+
%\erf\left[\frac{1-m}{\sqrt{2}s}\right]+
%\erf\left[\frac{m}{\sqrt{2}s}\right]\right]^{-1}\\
%\beta=&1-\frac{k}{\sqrt{2\pi}s}+\frac{k}{2}
%\left[\erf\left[\frac{1-m}{\sqrt{2}s}\right]+
%\erf\left[\frac{m}{\sqrt{2}s}\right]\right]\\
%\mu=&\frac{1}{\beta}\left[\frac{1}{2}\left[1-\frac{k}{\sqrt{2\pi}s}\right]-
%\frac{ks}{\sqrt{2\pi}}\left[
%\exp\left[-\frac{(1-m)^2}{2s^2}\right]-\exp\left[-\frac{m^2}{2s^2}\right]\right]+
%\right.\\
%&\left.\frac{km}{2}\left[\erf\left[\frac{1-m}{\sqrt{2}s}\right]+
%\erf\left[\frac{m}{\sqrt{2}s}\right]\right]\right]\\
%\sigma^2=&\frac{1}{\beta}\left[\frac{1}{3}\left[1-\frac{k}{\sqrt{2\pi}s}\right]-
%\frac{ks}{\sqrt{2\pi}}\left[(1+m)\exp\left[-\frac{(1-m)^2}{2s^2}\right]-
%m\exp\left[-\frac{m^2}{2s^2}\right]\right]+\right.\\
%&\left.k\frac{m^2+s^2}{2}\left[\erf\left[\frac{1-m}{\sqrt{2}s}\right]-
%\erf\left[\frac{m}{\sqrt{2}s}\right]\right]\right]-\mu^2
%\end{align}

\section{Tests}

\subsection{Particle kinematics}

\begin{figure}
\centerline{\includegraphics[width=\columnwidth]{kinetic}}
\caption{Phase space}
\label{fig:kinetic}
\end{figure}

\subsection{Energy conservation}

\begin{figure}
\centerline{\includegraphics[width=\columnwidth]{energy2d}}
\caption{Phase space}
\label{fig:energy2d}
\end{figure}

\begin{figure}
\centerline{\includegraphics[width=\columnwidth]{energy3d}}
\caption{Phase space}
\label{fig:energy3d}
\end{figure}

\bibliography{abbrevs,research,books}

\clearpage

\appendix
\section{List of \picsim options}

The options are described by classes.
\begin{itemize}
\item \emph{PIC} class
	\begin{itemize}
	\item -h, --help
	\item -v, --version
	\item -i, --input [filename]
	\item Background
	\item Beam
	\item DrivenBackground
	\end{itemize}
\item \emph{NumberGenerator} class
	\begin{itemize}
	\item -h, --help
	\item -v, --version
	\item -i, --input [filename]
	\item seed, -seed
	\end{itemize}
\item \emph{Scheduler} class
	\begin{itemize}
	\item -h, --help
	\item -v, --version
	\item -i, --input [filename]
	\item timeinc 
	\item maxiter
	\item refmass
	\item reftemp
	\item gridsize
	\item rmin
	\item rmax
	\item eforce
	\item emode
	\item evector
	\item bforce
	\item bvector
	\item gforce
	\item gvector
	\end{itemize}
\item \emph{Backgrd} class
	\begin{itemize}
	\item -h, --help
	\item -v, --version
	\item -i, --input [filename]
	\item \emph{speciesname}:mass	
	\item \emph{speciesname}:charge	
	\item \emph{speciesname}:lambda	
	\item \emph{speciesname}:drift
	\item \emph{speciesname}:temp
	\item \emph{speciesname}:bcmin
	\item \emph{speciesname}:bcmax
	\item \emph{speciesname}:rmin
	\item \emph{speciesname}:rmax
	\item \emph{speciesname}:hydrostatic
	\item \emph{speciesname}:imodemin
	\item \emph{speciesname}:imodemax
	\end{itemize}
\item \emph{Beam} class
	\begin{itemize}
	\item -h, --help
	\item -v, --version
	\item -i, --input [filename]
	\item \emph{speciesname}:mass	
	\item \emph{speciesname}:charge	
	\item \emph{speciesname}:lambda	
	\item \emph{speciesname}:drift
	\item \emph{speciesname}:temp
	\item \emph{speciesname}:bcmin
	\item \emph{speciesname}:bcmax
	\item \emph{speciesname}:rmin
	\item \emph{speciesname}:rmax
	\item \emph{speciesname}:beamdir
	\item \emph{speciesname}:beampos
	\item \emph{speciesname}:seff
	\item \emph{speciesname}:depth
	\end{itemize}
\item \emph{DrivenBackgrd} class
	\begin{itemize}
	\item -h, --help
	\item -v, --version
	\item -i, --input [filename]
	\item \emph{speciesname}:mass	
	\item \emph{speciesname}:charge	
	\item \emph{speciesname}:lambda	
	\item \emph{speciesname}:drift
	\item \emph{speciesname}:temp
	\item \emph{speciesname}:bcmin
	\item \emph{speciesname}:bcmax
	\item \emph{speciesname}:rmin
	\item \emph{speciesname}:rmax
	\item \emph{speciesname}:hydrostatic
	\item \emph{speciesname}:imodemin
	\item \emph{speciesname}:imodemax
	\item \emph{speciesname}:fmin
	\item \emph{speciesname}:fmax
	\item \emph{speciesname}:amin
	\item \emph{speciesname}:amax
	\item \emph{speciesname}:posmin
	\item \emph{speciesname}:posmax
	\item \emph{speciesname}:devmin
	\item \emph{speciesname}:devmax
	\item \emph{speciesname}:perturbmin
	\item \emph{speciesname}:perturbmax
	\item \emph{speciesname}:startmin
	\item \emph{speciesname}:startmax
	\item \emph{speciesname}:stopmin
	\item \emph{speciesname}:stopmax
	\end{itemize}
\item \emph{PoissonSolver} class
	\begin{itemize}
	\item -h, --help
	\item -v, --version
	\item -i, --input [filename]
	\item gridsize
	\item bcmin
	\item bcmax
	\item rmin
	\item rmax
	\item tolmax
	\item delta
	\item maxcy
	\item kcycle
	\item ipre
	\item ipost
	\item intpol
	\item temin
	\item temax
	\item stdte
	\item teshape
	\item qnc
	\end{itemize}
\end{itemize}


\end{document}
